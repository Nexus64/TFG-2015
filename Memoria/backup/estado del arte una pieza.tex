\chapter{Estado del arte}
\label{chap:estado del arte}
\section{El mercado de los videojuegos}
El videojuego es la industria de ocio y entretenimiento líder tanto en ventas como en crecimiento.

En el año 2015, el mercado del videojuego genero unas ventas de 91.800 millones de dolares, superando el doble de lo producido por la industria cinematográfica (38.300 millones de dólares) y en más de seis veces al mercado de la música (15.000 millones de dolares)~\cite{libro_blanco} según los datos de la Federación Internacional de la Industria Fonográfica.

Los datos de esta sección han sido obtenidos del Libro blanco del desarrollo del videojuego~\cite{libro_blanco}, informe desarrollado por DEV, la asociación española de empresas desarrolladoras de videojuegos.

\subsection{Industria mundial del Videojuego}
El videojuego es actualmente la industria de ocio y entretenimiento líder en ventas y en crecimiento. Superando en dos veces y media la dimension del mercado del cine y en mas de seis veces el de la música, e mercado mundial de los videojuegos generó en 2016 unas ventas de 101.1 millones de dólares. El mercado global del videojuego seguirá creciendo con una tasa anual del 6,6\%, hasta llegar en 2019 a los 118.600 millones de dólares.

Se estima que, en todo el planeta, haya más de 1.500 millones de jugadores, aunque de entre ellos solo un pequeño 300 millones de usuarios generaría el grueso de las ventas directas, mientras que el resto juegan realizando gastos mucho menos. La mayor parte de los jugadores juegan en plataformas mobile (75\%) y en PC (73\%), siendo las consolas el dispositivo utilizado solamente por el 14\% de los usuarios. Esta diferencia se hace aún más evidente en los países asiáticos y en menor medida en Europa, mientras que en Estados Unidos el uso de las distintas plataformas es más homogéneo.

De entre las distintas plataformas de juegos, segmento con mayor peso en el mercado global en el año 2017 será el mercado de los smartphones, con unos ingreso anual de 46.100 millones de dólares, seguido por las consolas de sobremesa, con 33.500 millones de dólares. El segmento mobile, combinación del segmento de los smartphones con el pequeño segmento de las Tablets, es el que crecerá de forma más contundente, llegando a representar la mitad del total global – unos 64.900 millones de dólares – en 2020.

Si se observan los ingresos generados en las distintas regiones geográficas, Asia Pacífico lidera el mercado con una previsión de 51.200 millones de dólares en 2017, lo que representa el 47\% del total de los ingresos globales del juego, seguido en segunda posición por mercado norteamericano con 23.500 millones de dólares en 2016, representado prácticamente en su totalidad por Estados Unidos. Más de la mitad de los ingresos de Asia Pacifico son generados en China, con 227.500 millones de dólares.Europa se posiciona con 5 países entre los 10 mayores mercados mundiales por tamaño de ingresos totales: Alemania, Reino Unido, Francia, España e Italia. Sin embargo, la diferencia en tamaño con respecto a los tres primeros mercados – China, Estados Unidos y Japón – es abismal. 

Por otro lado, si se tiene en cuenta el gasto medio anual de aquellos consumidores dispuestos a pagar, los jugadores más lucrativos resultan ser los japoneses y surcoreanos (alrededor de 300\$ de gasto medio), seguidos por los canadienses y estadounidenses (200\$ de gasto medio). China se posiciona entre los países con menor gasto medio por jugador dispuesto a pagar, pero el potencial de crecimiento en términos de poder de adquisición y de número de jugadores es enorme – menos de un tercio de la población juega a videojuegos, frente a una media del 50\% en los demás países occidentales y asiáticos.

Considerando únicamente el segmento de videojuegos para dispositivos móviles, la distribución de las ventas por mercados geográficos no varía mucho con respecto al total. La región Asia-Pacífico representa más de la mitad del mercado mobile mundial y es liderada por China que mueve 6.500 millones de dólares y crece anualmente del 46,5\%. Las regiones que están experimentando mayor crecimiento son el sureste asiático, con un 69\% anual, Latinoamérica, con un 73\% anual y Africa-Oriente Medio, con un impresionante 83\%.

El mercado global de los videojuegos está muy polarizado, siendo a las 25 mayores empresas las que generaron el 67\% de las ganancias globales, sumando un total de 61.600 millones de dólares en 2015. Tencent, que fue la compañía con mayores ingresos en 2015 (8.700 millones de dólares) supera en dos veces el tamaño del mercado en Corea del Sur en su totalidad y en casi 5 veces el español. Por otro lado, las consolidaciones en el sector – como las recientes compras de King por Activision-Blizzard y la de Supercell por Tencent – están contribuyendo a aumentar el peso de los gigantes del sector.

\subsection{La industria de los videojuegos en España}
En nuestro pais, la industria del videojuego del videojuego mantiene un ritmo alto de crecimiento, habiéndose incrementado en un 20\% en el ultimo año a pesar de la pasada crisis económica. Actualmente, el sector cuenta con un total de  480 empresas en activo, a las que debemos añadir las 125 iniciativas y proyectos empresariales, que se encuentran a la espera de consolidarse como empresas en el corto o medio plazo.

Este crecimiento ha sido muy acelerado en el ultimo decenio. De las empresas actuales, el 85\% se fundaron hace menos de 10 años, y el 63\% tienen menos de 5 años. El periodo de crecimiento coincide con la explosión del mercado de los videojuegos para dispositivos móviles a nivel mundial y, en menor medida, la aparición de servicios de distribución digital, como Steam. Estos nuevos mercados fomentaron la aparición de pequeños estudios independientes de plantilla reducida.

Los principales centros de actividad del país son las Comunidades de Madrid y Cataluña, donde se concentra la mitad de los estudios de videojuegos nacionales. A las dos primeras comunidades les sigue la Comunidad Valenciana, que cuenta con un 11\% de las empresas del sector, Andalucía, con un 9\% y el País Vasco, con un 7\%.

La industria del videojuego en españa está formada mayoritariamente por empresas pequeñas, ya que el 95\% de las empresas tiene una platilla de menos de 50 empleados. Sin embargo, el sector se encuentra muy polarizado: las microempresas (aquellas con menos de 10 empleados), que suponen el 71\% de la masa empresarial, solamente generan el 20\% del empleo nacional, mientras que las medianas y grandes empresas (más de 50 empleados), a pesar de suponer solo el 8\% del total, emplean a casi la mitad (48\%) de los profesionales del sector.

La industria del videojuego es altamente creativa y generadora de nueva propiedad intelectual. La mayor parte de las empresas se dedica al desarrollo de IP propia o desarrollo para terceros. Es relevante destacar que muchas empresas se dedican a múltiples actividades, es decir, compaginan el desarrollo de IP propia con el desarrollo para terceros y la formación. Con respecto a 2015, se ha detectado un ligero descenso de estudios que desarrollan IP propia (de 82\% a 76\%).

La industria de desarrollo de videojuegos facturó 510,7 millones de euros en 2015, un 24\% más que en el ejercicio anterior. El 84\% de la facturación se reparte en igual medida entre los dos grandes polos empresariales de este sector, representados por la Comunidad de Madrid y Cataluña. Les sigue Andalucía que representa un 8\% del total de la facturación nacional. Las sedes de los estudios de más envergadura y, por tanto, de mayor facturación, están localizadas en estas regiones.

La tendencia a la polarización que se observa en el empleo, se presenta de forma aún más acentuada en la distribución de la facturación. El 1\% de las empresas concentra el 52\% de la facturación. Las microempresas (facturación menor de 2 millones de euros) representan el 83\% de la masa empresarial, pero solamente el 8\% de la facturación. El 23\% de las empresas españolas, al encontrarse todavía en fases de desarrollo de sus respectivos proyectos, todavía no han iniciado su actividad comercial y, por tanto, no han empezado a facturar.

En cuanto a la procedencia de la facturación, se nota una bajada importante del peso de las ventas de juego en formato físico, cubriendo solo un 6\% de los ingresos. Por otro lado, el 34\% de la facturación proviene de la venta directa digital, mientras que el modelo "free to play" representa el 22\%. El casi 40\% restante de la facturación de nuestras empresas proviene de fuentes alternativas, como el desarrollo para terceros, la venta de servicios y la formación. 

Desde un punto de vista internacional, España ocupa un lugar importante, siendo el cuarto mayor mercado a nivel europeo y el octavo mundial. Sin embargo, el peso de la industria española en el mundo no se corresponde con el de nuestro mercado, puesto que la mayor parte de los juegos consumidos han sido desarrollados en el extranjero. Esta situación contrasta con la situación de otros paises como Finlandia, Suecia, Alemania o Francia, donde se obtiene una facturación muy superior a pesar de contar con menos empresas.

La venta internacional de videojuegos es la mayor fuente de facturación, suponiendo el 52\% del total. Esto se debe a la proliferación de los modelos de distribución digitales y la paulatina pérdida de peso de la distribución física. Norteamérica, con el 22\% y Europa, con un 19\%, representan  los mercados internacionales con mayor peso en la facturación de nuestras empresas, mientras que los mercados asiático y  suramericano representan menos del 10\%. La globalidad de nuestra industria se refleja en la disponibilidad de idiomas en las producciones españolas. El 97\% de las producciones están disponibles en inglés, idioma que supera al castellano (95\%). Les siguen otros idiomas europeos, como el francés y el alemán, pero con menos incidencia. 

A pesar del estancamiento generalizado de la creación de empleo en el país, la industria del videojuego consiguió incrementar su plantilla un 32\% en el año 2015. En total hay 4.460 profesionales en la industria, entre empleados y colaboradores directos. Si incluimos las estimaciones de empleos que fueron generados de forma indirecta por la actividad de la industria, se alcanza un total de 7849 profesionales.

La comunidad autónoma con mayor peso en la industria es Cataluña, con el 38\% del empleo – sin duda, por la importante contribución de empresas como King y Social Point – y de la Comunidad de Madrid, con el 28\%. Juntas concentran el 66\% del empleo nacional. Le sigue Andalucía que representa un 10\% del total. Las sedes de los estudios de más envergadura y, por tanto, con mayor empleo están localizadas en estas comunidades. Según los datos recogidos, una característica relevante del empleo generado por el sector es su elevada tasa de estabilidad, dado que casi el 60\% de los contratos tiene carácter indefinido.

\subsection{Retos y tendencias actuales}
El videojuego ha sido y es una industria muy cambiante, que siempre ha intentado integrar las tecnologías más punteras, desde innovadores algoritmos de renderizado gráfico hasta exóticos dispositivos de interacción persona-ordenador.

A continuación, listaremos algunas de las tendencias que van a influir fuertemente en el mercado en los años venideros:

\subsubsection{eSports}
Los eSports - también llamados "deportes electronicos" - es el nombre por el cual se conocen las competiciones de videojuegos. En los eSports, los jugadores profesionales compiten entre ellos en diferentes categorías: disparos en primera persona, lucha, estrategia en tiempo real, MOBAs (Multiplayer Online Battle Arena)... La popularidad de este fenómeno ha llegado al punto en el que los principales torneos se celebran en grandes estadios, están retransmitidos en streaming por Internet e incluso están dotados con premios de grandes cantidades de dinero y que en ocasiones superan el millón de euros. Se trata de unos eventos de gran popularidad que cuentan enormes con perspectivas de crecimiento (a una tasa anual del 40\%). Esto ha provocado que se haya posicionado como un fenómeno de ocio estratégico.

El mercado de los eSports, generó en 2015 325 millones de dólares de ingresos, creciendo del 67\% con respecto al año anterior. Si la situación se mantiene, se espera que la cifra supere los mil millones de dolares en el año 2019, manteniendo un crecimiento anual superior al 40\%. En 2015, hubo 230 millones de personas que vieron partidos de competiciones de diversos deportes electrónicos, de los cuales 115 millones pueden ser clasificados como “entusiastas”, es decir, espectadores regulares y participantes no muy distintos de los que encontraríamos en los deportes convencionales. La plataforma de retransmisión en directo Twitch registro más de 800 millones de horas de contenidos de eSports en el periodo entre agosto de 2015 y mayo de 2016.

Existen en España diferentes empresas que en la actualidad apuestan por el desarrollo de productos que aspiran a convertirse en eSports, es el caso de Digital Legends, Mercury Steam, PixelCream Studio, Mechanical Boss, entre otros. Sin embargo, en la mayoría de las ocasiones un videojuego online se convierte en eSports de manera orgánica, basada en el reconocimiento que la comunidad online de jugadores activos de ese videojuego le otorga. Existen eso si casos en las que una gran marca apuesta porque su producto se convierta en un eSports, realizando grandes inversiones en en base a una gran inversión en infraestructura, personal dedicado a la comunidad, servidores escalables para una gran masa de jugadores, premios para los torneos, entre otras muchas. 

Sin embargo, esta inversión no siempre esto asegura que su producto se convierta en un éxito. Para que un videojuego pueda convertirse en un eSport necesita contar con características básicas : tener un fuerte factor de competición, partidas cortas de no más de 1 hora, sin progresión in-game – la progresión se basa en las habilidades del jugador – atractivo sistema de espectador y tener un enfoque al 100\% internacional.

Pese a su gran dificultad, conseguir posicionar un producto como eSports, aporta una serie de beneficios y posibilidades: 
\begin{itemize}
\item Crear una base de fans, una comunidad, algo que aporta un núcleo de consumidores fieles al producto y que le da una nueva dimensión social, muy atractiva para muchos de los consumidores de videojuegos.
\item Prolongar la vida del producto; al ser competitivo, el jugador fija sus metas ante los otros jugadores, esto incentiva al usuario y le proporciona una motivación para seguir consumiendo.
\item Proporcionar mayor visibilidad, ya que, aunque los productos asentados son extremadamente sólidos, son muy reducidos, por lo cual hay una demanda latente de usuarios que buscan nuevos eSports.
\item Aumentar la fidelidad de los usuarios al tratarse de un mercado donde los usuarios tienen un índice de fidelidad mucho más alto que en otros.
\item Los jugadores, al estar involucrado con un producto competitivo, ven streaming, leen noticias, siguen torneos, participan en foros, lo que disminuye el riesgo de abandono del producto.
\end{itemize}

Para una empresa pequeña, la producción de un eSports es, en principio inabarcable, inabarcable debido a la elevada inversión mencionada anteriormente. Sin embargo, la asociación con unos los partners correctos, ya establecidos en el sector, que pudieran invertir en el producto y le den visibilidad, puede dar una gran oportunidad a los pequeños desarrolladores que cuentan con una mayor flexibilidad con respecto a las grandes compañias, lo que les permite adaptarse más rápidamente a un mercado tan cambiante y mantener una relación directa con el feedback del usuario.

\subsubsection{Realidad Virtual y Realidad Aumentada}
La realidad virtual (normalmente abreviada como VR por las siglas inglesas de Virtual Reality) es la tecnologia generada por sistemas informáticos que proporcionan un entorno audiovisual en 3D el que el usuario puede experimentar una inmersión total. Para ello, se hacen usos de cascos especiales equipados con pantallas y sensores de movimiento, los cuales normalmente se complementan con mandos equipados también con sensores para permitir una interacción más natural.
Por otro lado, la Realidad Aumentada o AR es una tecnología que superpone una capa de gráficos generados por ordenador sobre el entorno que rodea al jugador, con la que este puede interaccionar en tiempo real. A diferencia de la VR, no se requiere obligatoriamente de un hardware especial para poder implementar AR; basta únicamente de un dispositivo equipado con una pantalla y una cámara de vídeo, como podría ser un Smartphone.

Actualmente, existen varias propuesta de diversas compañías en lo que a equipo de VR se refiere. Vamos a mencionar algunas de las más importantes:
\begin{itemize}
\item \textbf{HTC Vive:} es la propuesta de HTC y Valve, orientada a jugadores "hardcore" de PC. Disponible desde abril de 2016, el dispositivo requiere de un PC de gama alta (Valve recomienda un PC con una gráfica GeForce GTX 970). El kit de hardware incluye el casco equipado con dos pantallas de 1080x1200 puntos y 90Hz de frecuencia de actualización, dos sensores espaciales y dos mandos para registrar los movimientos de ambas manos, lo que crea le permite crear un entorno 100\% virtual en el que sumergir al jugador.

\item \textbf{OCULUS Rift:} Es la propuesta más veterana de la lista. Empezó como un exitoso proyecto de Kickstarter en 2012 que más tarde fue adquirida por la empresa Facebook dos años más tarde. Al igual que HTC Vive, Oculus está formado por un casco equipado con pantallas de alta resolución, mandos con sensores de movimiento y dos sensores de posición. El equipo necesita estar conectado a un PC de alta gama para poder funcionar correctamente.

\item \textbf{Samsung Gear VR:} La propuesta de Samsung es mucho más sencilla y económica, orientado más a la reproducción de vídeo en 360º (concepto similar a la realidad virtual pero con interactividad limitada). El casco incluye una unica pantalla y sus mandos carece de detección de movimiento. Estas limitaciones conllevan, por otro lado, un precio mucho más accesible que el de las otras alternativas (99€ contra los mas de 500€ de las propuestas más completas)

\item \textbf{SONY PlayStation VR:} la propuesta de Sony fue lanzada en el año 2016. Al igual que otras alternativas, el sistema se basa en un casco equipado con dos pantallas y sensores de movimiento, pero su principal punto de venta es su compatibilidad con la consola PlayStation 4 de la misma marca. Esto permite aprovechar la potencia y los mandos de control de está de la consola.
\end{itemize}

\subsubsection{Web 4.0}
El termino Industria 4.0 fue acuñado por el Ministerio de Educación y Desarrollo alemán en su plan estratégico de 10 puntos del año 2016 para mejorar la educación, investigación e industria del país para adaptarlas a las tecnologías de Internet. La estrategia trata cinco áreas principales:
\begin{itemize}
\item Fuerte cooperación entre la investigación científica y las empresas.
\item Aumentar la innovación en el sector privado.
\item Diseminar las tecnologías punteras.
\item Internacionalizar la investigación y desarrollo.
\item Fondos para individuos con talento.
\end{itemize}

De entre las distintas tecnologías que podrían categorizarse como parte de la industria 4.0, vamos a describir aquellas que tienen mayores aplicaciones en el desarrollo de videojuegos:

La computación en la nube, o Cloud Computing, es la tecnología que permite el acceso a servicios informáticos de forma rápida y sencilla a través de Internet. Aunque aun no se a podido implementar correctamente el Cloud Gaming (donde el juego es íntegramente ejecutado en la nueve, reduciendo la exigencia de potencia del sistema del jugador), si se utilizan sistemas en la nube en distintas áreas de los videojuegos. Especialmente notable es su uso para el control y almacenamiento de información en juegos multijugador en linea.

El Internet de las cosas es como se conoce a la tecnología que permite dotar de conexión a Internet a todo tipo de pequeños dispositivos como relojes, sistemas de domótica, drones, sensores de todo tipo, robots, etc. Esto permite implementar videojuegos en todo tipo de sistemas, desde consolas portátiles cada vez más pequeñas y económicas, pasando por juguetes interactivos y llegando a la posibilidad de gamificar con facilidad procesos industriales.

Big Data es el proceso de clasificar grandes volúmenes de datos para poder obtener relaciones interesantes y no evidentes entre ellos. El principal uso de las técnicas de BigData en la industria del Videojuego es el análisis de la información de los jugadores. Analizando datos de los jugadores tales como el género, la edad, la localización geográfica, los intereses, los gastos realizados, etc, es posible obtener estrategias de negocios eficientes.

Los sistemas Ciberfisicos son  un nuevo tipo de sistemas con unos componentes hardware y software estrechamente interconectados, cada uno operando en su propio ámbito, operando e interacionando de forma distinta dependiendo del contexto. Entre sus aplicaciones se encuentran las redes eléctricas inteligentes, los sistemas de conducción automática de aviones y automóviles o la monitorización médica. Las interfaces de estos sistemas requieren de unas interfaces con un fuerte "lado humano" que permita un uso sencillo e intuitivo. Aquí se podrían utilizar los principios de diseño de juego que permitirían desarrollar un mejor puente entre el lado máquina y la parte usuario.

La Impresión 3D, también conocida como la producción aditiva es una tecnología que permite producir objetos de forma más sencilla que con las técnicas anteriores. En combinación con las técnicas de escaneado 3D, los estudios de videojuego pueden generar de forma rápida y eficiente modelos 3D de todo tipo (personajes, mapas, objetos...).

Las nuevas tecnologías de la industria 4.0 serán de gran ayuda para el desarrollo de videojuego. Pero es posible que la industria 4.0 también ofrezca valor a la industria 4.0 en su conjunto. Dado que la creación de videojuegos es un actividad industrial que está vinculada a diferentes áreas de conocimiento que trabajan juntas para conseguir ofrecer un producto, las técnicas y paradigmas utilizados tienen mucho en común con la nueva forma de trabajar de la industria 4.0, por lo que es posible que puedan extrapolarse a otras industrias, permitiendo una mejor adaptación a los cambios.

\section{El proceso de desarrollo de videojuegos}
\subsection{Introducción}
El desarrollo de un videojuego, como el de cualquier otro producto software, debe de ser planificado correctamente y ejecutado siguiendo una metodología adecuada. Sin embargo, El diseño y desarrollo de un videojuego requiere de la participación de campos ajenos a la informática como el diseño de juegos, el diseño gráfico o la composición musical. Una parte importante de  la producción consistirá en organizar a un equipo multidisciplinario para poder terminar el proyecto dentro del tiempo y presupuesto acordados.

\subsection{Etapas del desarrollo}
\subsubsection{Desarrollo del Concepto}
El desarrollo de todo videojuego comienza con una idea. Durante la fase de desarrollo del concepto se tomará dicha idea para obtener una diseño preliminar listo para pre-producción. El objetivo principal de esta etapa es decidir sobre qué tratará el juego y ponerlo por escrito para que cualquier miembro del equipo lo pueda entender con claridad, decidiendo las principales mecánicas, creando el arte conceptual y escribiendo el argumento. 

Al final de la etapa se habran elaborado tres documentos: El "High Concept", el "Pitch Document" y el "Concept Document". El High Concept consiste en una o dos frases que describen a grandes rasgos como será el juego, en especial que lo hace distinto de la competencia. El Pitch Document, o Propuesta de Juego, es un pequeño documento de en torno a dos paginas, orientado a ser leído en las reuniones donde el juego sea propuesto a inversores. El documento resume las características del juego y explica por que será exitoso y rentable.

Finalmente, el Concept Document es un extenso documento que explica en detalle las características del juego. Se trata de una versión extendida de Pitch Document que trata temas como: 
\begin{itemize}
\item El High Concept
\item El género del juego
\item Características principales y jugabilidad.
\item Ambientación e historia.
\item Estimación del presupuesto y de la planificación.
\item Equipo de desarrollo.
\item Análisis de Riesgos.
\end{itemize}

\subsubsection{Pre-Producción}
La pre-producción es la fase de preparación: en la que el equipo diseña y planifica los elementos que serán desarrollados en la etapa de producción. Al final de esta etapa, se debe haber completado el diseño de juego, creado la biblia de arte, elaborado el documento de diseño técnico, establecido el plan de producción, y creado un prototipo del juego final.

El diseño del juego quedará establecido en un Documento de Diseño de Juego (o GDD, por sus siglas en ingles). Se trata de un documento "vivo", que está en constante modificación para adaptarse a los ajustes concretos de diseño que se realizarán durante el desarrollo.

La biblia de arte es una colección de arte conceptual que servirá para definir el estilo artístico del juego desde un primer momento. La biblia incluirá también una librería de imágenes de referencia que puedan ser de ayuda a los artistas que desarrollen los elementos gráficos finales.

El Documento de Diseño Técnico contiene una descripción en detalle la parte técnica del proyecto definiendo las tareas que los desarrolladores deberán afrontar, estimando el coste que dichas tareas tendrán tanto en tiempo como en número de personas y especificando las herramientas y técnicas que utilizará el equipo.

El Plan de producción recopila la información acerca de como se va a desarrollar el proyecto. Este incluye las tareas a realizar junto a los tiempos, costes y dependencias de estas, divididos en varios documentos menores para poder ser organizado mejor:
\begin{itemize}
\item \textbf{Plan de mano de obra:} Listado del personal, sus horarios y su salario.
\item \textbf{Plan de recursos} Estimación del coste los recursos externos al proyecto (música, arte, herramientas...)
\item \textbf{Documento de seguimiento:} Documento donde se realiza un control de los tiempos y plazos del proyecto.
\item \textbf{Presupuesto}: Contiene el coste mensual del proyecto y el calculo del presupuesto general
\item \textbf{Ganancias y Perdidas}: Estimación  de las ganancias y  perdidas del proyecto. Debe ir actualizándose según se avanza en el desarrollo.
\item \textbf{Definición de Hitos}: Lista de las distintas "metas" del proyecto, que son puntos del desarrollo donde se habrá terminado una cantidad de trabajo importante.
\end{itemize}

Una vez diseñado y planificado el proyecto, el equipo empezara a trabajar en la creación de un Prototipo. Un prototipo es una pieza de Software funcional que contiene una pequeña fracción del software final. El desarrollo prototipo servirá para varias funciones: poner a prueba el diseño de juego, concretando de forma más precisa la jugabilidad; realizar un simulacro de desarrollo para determinar las dinámicas del equipo y producir una muestra del juego final a inversores y publicadores.

\subsubsection{Producción}
La producción es la etapa principal del desarrollo del juego. Durante esta etapa se elaboraran e implementarán los elementos descritos y diseñados durante la pre-producción: los programadores implementarán los sistemas del juego, los artistas elaboraran los gráficos definitivos, los diseñadores crearán misiones y niveles, etcétera. Dependiendo del juego en cuestión, una producción normal suele durar entre seis meses y dos años, aunque el desarrollo de juegos pequeños para, por ejemplo, dispositivos móviles puede realizarse en menos tiempo aún.

La etapa de producción es de naturaleza iterativa: El juego va construyéndose en varias etapas en las que se implementan pequeñas porciones de este. Entre etapa y etapa, el juego pasa por un proceso de pruebas en la que se verifica su usabilidad y robustez frente a fallos. Los resultados de las etapas se utilizan como base para recalcular los tiempos y presupuestos de las etapas posteriores. Dividir el desarrollo de esta forma hace que sea más sencillo afrontar problemas y contratiempos que el equipo podría encontrar en las etapas tardías.

\subsubsection{Final de Producción y Lanzamiento}
Durante las ultimas etapas de la producción, el paradigma de desarrollo cambia, pasando el objetivo de implementar contenido nuevo a pulir y ajustar el contenido pre-existente. Esta parte de la producción puede dividirse en dos etapas: Alpha y Beta

La fase Alpha o Code-Complete es el punto del desarrollo donde el juego se encuentra en un estado jugable, a falta solo de ciertos vacíos como gráficos provisionales o minijuegos o sub-sistemas incompletos. El objetivo de esta etapa es el de encontrar y corregir todos los fallos posibles y también probar y ajustar la jugabilidad.

En la fase Beta o Content-Complete la mayor parte del contenido del juego deberá estar terminado y deben haber pocos o ningún fallo importante en el juego. En esta etapa el juego es puesto en manos de equipos de testing externos a la empresa para realizar análisis exhaustivos en busca de fallos que se le pueden haber escapado al equipo de testing interno. Es también en esta etapa donde la campaña de publicidad del juego deberá ser más fuerte.

Una vez superadas las dos etapas de pruebas, la Versión Final del juego es enviada a la distribuidora para que comience la producción de las copias físicas, o para que el juego aparezca en las plataformas de distribución.

\subsubsection{Post-Producción}
Una vez lanzado el juego, el equipo entra en la etapa de Post-Producción. En esta etapa el equipo trabajará en corregir fallos y problemas que los jugadores encontraron tras el lanzamiento y en la elaboración de contenido adicional descargable.

La duración y trabajo de la post-producción depende mucho del juego en cuestión. Hasta hace relativamente poco, los juegos lanzados en videoconsolas carecían completamente de esta etapa debido a la dificultad para modificar los juegos que ya ese encontrasen en el mercado. por otro lado, a día de hoy la mayor parte de los videojuego reciben parches y actualizaciones sin importar su plataforma de distribución.

Un caso especial sería el de los juegos con un fuerte componente Online, como los MMORPG, los MOVA o los shooter en linea. Los desarrolladores de este tipo de juegos, para mantener contenta a su base de jugadores y evitar el estancamiento, lanzan de forma periódica actualizaciones que ofrecen nuevo contenido, mejoras y ajustes. Algunos de estos juegos pueden recibir este tipo de actualizaciones durante años, como World of Warcraft (MMORPG de Blizzard, en activo desde 2004) o Team Fortress 2 (FPS de Valve, en activo desde 1999)

\subsection{Estructura típica de un equipo de desarrollo}
El desarrollo de un videojuego puede llevarse a cabo por equipos de desarrollo muy distinto dependiendo de la extensión del proyecto, desde una sola persona creando un pequeño juego indepeesteondiente, el cual realiza todo el trabajo por si mismo; hasta los equipos de cientos de personas que desarrollan los juegos "triple A", organizados en múltiples departamentos, cada uno con su estructura jerárquica.

A pesar de esto, existen una serie de roles que están presentes en todos los desarrollos. 
\subsubsection{Diseño}
La primera y más importante parte del desarrollo de un juego es el Diseño del mismo. El trabajo del diseñador es de describir el juego con un alto nivel de detalle, definiendo de con precisión todos las mecánicas, personajes, mapas, misiones del juego. Deberá también, hasta cierto punto, coordinar y dirigir el trabajo del resto de miembros del equipo para que puedan implementar correctamente el juego. 

En equipos grandes, el rol de diseñador puede dividirse en las siguientes categorías:
\begin{enumerate}
\item \textbf{Diseñador Jefe:} Es el encargado de dirigir al resto de diseñadores, decidiendo que contenido entra o no en el juego. Suele ser la persona que tuvo la "idea" original del juego.
\item \textbf{Diseñador de mecánicas:} El diseñador de mecánicas es el encargado de diseñar los distintos sistemas de juego, sirviendo como puente entre el diseñador jefe y los programadores. Debido a esto, el diseñador de mecánicas suele tener un trasfondo de programador.
\item \textbf{Diseñador de niveles:} También llamados Diseñadores de misiones, son los encargados de crear las distintas etapas que componen el juego, ya sean niveles, misiones, desafíos o puzles.
\item \textbf{Escritor:} La tarea del escritor es la de crear la historia del juego, así como la de escribir los distintos textos de este, como diálogos o descripciones. Se trata de una tarea muy distinta de la de un escritor de novelas o de guiones de película, ya que debe conciliar la narrativa con las exigencias de otros componentes del juego como el diseño o el arte.
\end{enumerate}

\subsubsection{Programación}
El rol del programador es el de implementar el juego en forma de código ejecutable. Esto supone el diseño e implementación de todo tipo de componentes imprescindibles: motor de renderizado, librerías para trabajar con sistemas de audio o conectarse por Internet, herramientas para integrar fácilmente el contenido artístico entre otras.

Cuando el equipo de desarrollo es grande, el rol de programador tiende a dividirse para cubrir tareas más especificas:
\begin{enumerate}
\item \textbf{Programador Jefe:} El programador jefe es frecuentemente el programador con más experiencia del equipo y el es el encargado de resolver las tareas más complicadas e importantes del proyecto. Cuando el equipo es muy grande, suelen realizar también tareas de coordinación.
\item \textbf{Programador de Mecánicas:} Es el encargado de convertir el diseño de juego en código ejecutable. Entre sus tareas se encuentra definir las físicas del mundo del juego, definir las funciones de los distintos objetos y modelar el comportamiento de los personajes.
\item \textbf{Programador de gráficos 3D:} Es el responsable de implementar los sistemas para la creación y renderizado de gráficos 3D. El programador de gráficos 3D necesita contar con conocimientos avanzados en calculo, matemática vectorial y matricial, trigonometría y álgebra.
\item \textbf{Programador de Interfaces:} Es el encargado de implementar los sistemas de interacción entre el jugador y el juego, normalmente interfaces de control, menús y HUDs (Head-Up Displays. 
\item \textbf{Programador de Audio:} Es la persona responsable de la implementación de los distintos sistemas que se utilizarán para reproducir música y sonido en el juego
\item \textbf{Programador de Herramientas:} El programador de herramientas tiene la responsabilidad de crear las distintas herramientas que el resto del equipo pueda necesitar para realizar, o acelerar, su trabajo. Un tipo especializado de programador de herramientas es el programador del editor de niveles, debido a la importancia de esta herramienta para el desarrollo del juego y por la posibilidad de que dicho editor sea lanzado al publico como parte del juego.
\item \textbf{Programador de Red:} Es el encargado de escribir el código que permite a los juegos ser ejecutados entre varios equipos, ya sea código maquina de bajo nivel o la integración de un librería de alto nivel.
\end{enumerate}

\subsubsection{Arte}
Se denomina artista a la persona o grupo encargado de generar los componentes gráficos del juego: modelos 3D de personajes y objetos, texturas, diseño de menús e interfaces, bocetos, animaciones y demás.

Existen varias categorías de artistas distintas dependiendo de en que rama se especialicen y de cual sea su rol en la estructura del proyecto:
\begin{enumerate}
\item \textbf{Director Artístico:} Asignado al artista con mayor experiencia en la industria, el papel del director artístico es el de organizar y coordinar al resto de artistas para que realicen correctamente su trabajo y el de revisar las piezas producidas para asegurarse de que son consistentes con el estilo artístico establecido.
\item \textbf{Artista Conceptual:} El artista conceptual es el encargado de producir bocetos provisionales que servirán como base para construir los gráficos definitivos del juego.
\item \textbf{Artista 2D:} Los artistas 2D son expertos en las técnicas tradicionales del dibujo y pintura.  Su rol es más notable en los juego 2D, donde deben producir la mayoría de los componentes gráficos como fondos, "tiles" y "Sprites"; pero también tienen un papel notable en los juegos 3D, donde suelen ser los encargados de diseñar interfaces gráficas, crear las texturas de los modelos 3D e incluso realizar trabajos ajenos al propio juego como la creación de imágenes promocionarles.
\item \textbf{Modelador 3D:} Es el encargado de producir los modelos 3D de los distintos componentes del juego tales como personajes, objetos, mapas...  Es relativamente común que los modeladores 3D tengan ciertos conocimientos de programación debido a que eran necesarios para trabajar con los primeros programas de modelado 3d.
\item \textbf{Animador:} Es el encargado de animar los diferentes elementos del juego, desde el simple movimiento de un molino de viento hasta las complicadas expresiones de una cara. Existen dos alternativas para realizar la animación: la técnica de keyframing, que consiste en realizar poses estáticas de los personajes que el programa utiliza para generar la animación; y la captura de movimiento, en la que los movimientos de un actor son capturados y transferidos al juego mediante un equipo especializado.
\item \textbf{Storyboarder:} Es el artista encargado de diseñar escenas del juego. Para ello, el Storyboarder crea unas secuencias de arte conceptual que describen los tiempos, diálogos y eventos de las escena, lo que permite valorarla y validarla antes de iniciar el costoso proceso de producción.
\end{enumerate}

\subsubsection{Audio}
El trabajo de audio en un videojuego viene en tres categorías: música, efectos de sonido y doblaje. Existen especialistas que se dedican exclusivamente a una sola de estas categorías, aunque no es raro encontrarse en pequeños estudios a una persona encargarse tanto de la música como del sonido. Reflejando los tipos de audio, los tres tipos de profesionales son:
\begin{enumerate}
\item \textbf{Músico:} Es el artista encargado de escribir las composiciones musicales que se escucharán a lo largo del juego. Es muy común que el músico también se haga cargo de interpretar sus composiciones mediante programas de síntesis de música, aunque las grandes producciones pueden permitirse contratar interpretaciones en vivo.
\item \textbf{Técnico de sonido:} Los técnicos de sonido son profesionales que se dedican a fabricar o adaptar sonidos para el proyecto en el que trabajen.
\item \textbf{Equipo de doblaje:} El trabajo de doblar un videojuego requiere del trabajo de varios profesionales. En primer lugar está el actor de voz, un actor especializado que interpreta con su voz a uno o varios personajes del juego. El trabajo de los actores está supervisado por un director de doblaje, que además suele encargarse de adaptar el guión y de dirigir a los técnicos de sonido que van a grabar y manipular las voces.
\end{enumerate}

\subsubsection{Testing}
El Aseguramiento de la calidad (o QA por sus siglas en inglés) es un requisito clave para el desarrollo de un videojuego, a la vez de un proceso lento y costoso que debe comenzarse lo más pronto posible para evitar un sobrecoste. El trabajo del del tester es el de revisar las distintas versiones del juego en busca de fallos para que los desarrolladores puedan arreglarlos.

Normalmente, los testers de un juego se agrupan en equipos, dirigidos por un jefe de testing. Cada equipo de testers se encarga de revisar una faceta distinta del juego, el equipo principal se encarga de probar la jugabilidad y los modos de juego individuales, el equipo multijugador se ocupa de revisar la jugabilidad en linea, así como los componentes técnicos de las conexiones, el equipo de compatibilidad prueba el juego en diversas plataformas y PCs con distintos componentes y el equipo de localización comprueba que se halla realizado una correcta traducción a distintos idiomas.

Para evitar la perdida de punto de vista critico que el equipo principal y el equipo de multijugador pueden sufrir tras haber trabajado con el juego desde el principio del desarrollo, es normal cambiar los equipos en las últimas etapas del desarrollo por equipos nuevos que no estén involucrados en el juego.

Junto al trabajo de los equipos profesionales de testing se suelen realizar campañas de Beta Testing. El Beta testing consiste en liberar una versión incompleta del juego para que jugadores aficionados los prueben. Las resultados y opiniones de los jugadores son recogidos para utilizarse en el desarrollo de la versión completa. Para realizar una exitosa campaña de beta testing es necesario contar con uno o más organizadores que puedan gestionar la retroalimentación de los usuarios.

\subsubsection{Producción}
La función principal del productor es servir de puente entre el equipo de desarrollo y el resto de la empresa. El productor debe tener un conocimiento profundo del juego y de los demás miembros del equipo, de forma que pueda explicarlo de forma correcta en las mochas reuniones que se tendrán con otros departamentos, como por ejemplo el de marketing.

El productor se encarga de realizar la gestión del proyecto, coordinando al equipo, realizando la programación de las etapas del proyecto y gestionando los posibles riesgos.

Existen tres tipos de productores dependiendo de su especialidad. Estos son:
\begin{enumerate}
\item \textbf{Productor Externo:} Este tipo de productor trabaja para la compañía editora y se encarga de supervisar al equipo de desarrollo para asegurarse de que se cumplen los acuerdos establecidos por ambas partes.
\item \textbf{Productor Interno:} Esta clase de productor trabaja en la compañía desarrolladora y se encarga tanto de realizar una gestión interna del proyecto como de actuar de representante de del equipo.
\item \textbf{Asistente de Producción:} Los asistentes de producción se encargan de realizar las tareas a las que el productor jefe del proyecto no puede dedicarse personalmente. Normalmente se trata administrar detalles concretos como administrar recursos, realizar el papeleo o gestionar los servidores y la página web.
\end{enumerate}

\section{Motores de juegos}
\subsection{Unity3D}
Unity3D, desarrollado en el año 2005 por David Helgason, Joachim Ante y Nicholas Francis, es un entorno integrado de desarollo para la creación de aplicaciones multimedia interactivas, típicamente Videojuego. 
Unity fue desarrollado por David Helgason, Joachim Ante y Nicholas Francis en el año 2005.

Unity incluye las herramientas necesarias para el desarrollo, junto con la tecnología necesaria para ejecutar Gráficos, Audio, Lectura de Entrada y Networking. 

La principales ventajas de Unity 3D frente a sus competidores es su facilidad para importar Assets, como modelos, sonidos o texturas, sin importar con que aplicación fueron creadas y por su capacidad para exportar a gran cantidad de plataformas distintas.

\subsection{Unreal Engine}
El Unreal Engine es un motor de juego desarrollado por la compañía Epic Game. Fue Desarrollado originalmente en el año 1998 para ser utilizado en el juego Unreal.

Aunque está diseñado principalmente para desarrollar juegos de tiros en primera persona, el motor es lo bastante flexible como para desarrollar juegos de cualquier genero, y en su versión actual incluye motor de renderizado, motor de audio, sistema de Inteligencia artificial avanzada y herramientas de animación integradas, entre otros.

Su código está escrito íntegramente en C++, lo que permite exportar los juegos a gran cantidad de plataformas distintas.

\subsection{Game Maker Studio}
Game Maker Studio es un motor de juegos desarrollado por Yoyo Games en el año 1999 con el objetivo de crear un motor de juegos accesible para personas sin conocimientos informáticos.

El motor cuenta con un lenguaje propio (Game Maker Language) de fácil aprendizaje y un sistema de programación gráfico basado en iconos. Sin embargo, su facilidad de uso reduce enormemente su potencia: el motor da soporte limitado a juegos 3D, carece de herramientas avanzadas para el desarrollo y tiene problemas para gestionar proyectos de gran tamaño.

A pesar de sus problemas, la simplicidad de Game Maker Studio lo hacen ideal para proyectos de pequeño tamaño o prototipos rápidos.

\section{Inteligencia Artificial en Videojuegos}
\subsection{Historia}
Desde los principios de la inteligencia artificial se han diseñado sistemas que puedan jugar a juegos contra adversarios humanos. 

El primer programa informático capaz de jugar a un juego a nivel respetable fue desarrollado por el ingeniero Arthur Samuel en el año 1956. El programa, que había sido diseñado para jugar al juego de la Damas, se basaba en  el algoritmo de búsqueda "MinMax" y en técnicas de aprendizaje automático. A pesar de las limitaciones de memoria (el programa se ejecutaba en un IBM 701) el programa podía jugar a un nivel intermedio.

La primera Inteligencia artificial surgió al año siguiente en 1957, desarrollado por Alex Berbstein. Este primer programa simulaba un oponente muy rudimentario y no fue hasta el año 1988 cuando el ordenador Deep Thought pudo superar a un gran maestro del ajedrez.

La inteligencia artificial tardó en ser implementada en videojuegos debido una vez más a las limitaciones técnicas. Los primeros juegos utilizaban simples patrones de comportamiento para los oponentes, técnica aun utilizada en la actualidad. Sin embargo, hubieron algunos juegos que implementaron sistemas más complejos, como el famoso Pacman (1980), donde el movimiento de los fantasmas enemigos está controlado por múltiples factores.

\subsection{Contexto Actual}
Existe una característica importante que distingue la Inteligencia Artificial utilizada en los videojuego de otros campos de aplicación. Mientras que la Inteligencia Artificial de "Verdad" trata en ultima instancia de obtener una "Inteligencia Fuerte" capaz de razonar y actuar de forma lógica, el objetivo de la Inteligencia Artificial dedicada a videojuegos es proporcionar una buena experiencia de juego al jugador, por lo que utiliza técnicas y estrategias mucho más sencillas y escalables.

Una de las modelos más utilizados para construir inteligencias artificiales en videojuegos son las máquinas de estados finitos. Este tipo de estructura de datos permite organizar el comportamiento de entidades como "estados", que permiten encapsular los distintos comportamientos y las transiciones entre ellos sin temor a "conflictos" entre distintas acciones. Las maquinas de estado son muy fáciles de implementar y tienen muchos usos, por lo que son ideales para implementar inteligencias artificiales básicas.

Los algoritmos de búsqueda de caminos son también una herramienta habitual en el desarrollo de inteligencias artificiales. La implementación de estos algoritmos es un sistema bastante complejo, debido a las limitaciones de CPU y memoria a las que están impuestas, sumado a la necesidad de realizar los cálculos en tiempo real para no estropear la experiencia del jugador. 

\subsection{Futuros campos de aplicación}
En la actualidad, la Inteligencia artificial se utiliza mayormente para el control de NPCs (Personajes no Jugables). Sin embargo, están empezando a surgir usos alternativos de los procesos de la inteligencia artificial~\cite{libro_blanco}.

En primer lugar tenemos la Generación Procedimental de Contenido (PCG, por sus siglas en inglés), nombre por el que se le conoce al conjunto de técnicas que permiten generar de manera automática contenido para el juego como, por ejemplo, mapas, reglas o misiones. Aunque las técnicas de PCG se llevan utilizando desde los años 80s, la aparición reciente de juegos exitosos como Minecraft ha levantado el interés de la comunidad de desarrolladores por ellas. Los beneficios principales del uso de PCG son la reducción del esfuerzo dedicado a la creación manual de contenido y la capacidad de adaptar el juego a las preferencias concretas del jugador.

Otro campo de estudio interesante es el Modelado de Experiencias de Jugado, el estudio y uso de técnicas de inteligencia artificial para la construcción de modelos computacionales de las experiencias del jugador. Estos modelos pueden ser utilizados para predecir determinados aspectos de la experiencia de juego (diversión, suspense, frustración...) de un jugador concreto, o de un grupo de jugadores, basándose en datos de entrada como el diseño de niveles, el estilo de juego del jugador o la velocidad del juego. Los beneficios del modelado de experiencias de jugador van desde la capacidad de valorar automáticamente la "calidad" de la experiencia entre varias versiones del mismo juego hasta la posibilidad de, en conjunto a la Generación procedimental de contenido, crear contenido adaptado específicamente a los gustos del jugador.