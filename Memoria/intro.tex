\chapter{Introducción}
\label{chap:introducción}
\section{Contexto}
%comparativa con otras industrias
La industria del videojuego es el \textbf{principal mercado del ocio y entretenimiento del mundo}, el cual engloba docenas de disciplinas de trabajo y da trabajo a miles de personas alrededor del mundo. Con unas ventas de 116.000 millones de dólares\cite{libro_blanco} en el año 2017, este mercado ha superado creces a sus competidores más cercanos: sus ganancias son dos veces y media superiores a las del mercado del cine y más de seis veces las del mercado de la música.

%industria en crecimiento
El éxito de esta industria se debe principalmente a su gran \textbf{capacidad para evolucionar} rápidamente para adaptarse a los gustos y tendencias de los consumidores. Esto ha permitido que los videojuegos hayan podido integrarse con las principales tendencias en tecnología, como lo fue en su momento la aparición de Internet o de los dispositivos móviles o como actualmente está ocurriendo con la realidad virtual o los servicios de distribución en línea.  

%Ampliación del público; creadores de contenido
La expansión del mercado del videojuego conlleva también una \textbf{expansión del perfil de sus jugadores}. La aparición de nuevos modelos de juego, como los juegos casuales, ha permitido llevar el mercado a un público que no probaría otros juegos más tradicionales. Adicionalmente, hay que incluir a los usuarios que utilizan los videojuegos para \textbf{crear su propio contenido}, como videos o modificaciones de juegos, o los que disfrutan como espectadores viendo jugar a jugadores profesionales.  

%Conclusión/En resumen...
En resumen, la industria del videojuego no solo es uno de los \textbf{mayores mercados actualmente}, también es un sector económico puntero en constante evolución y la forma de vida de millones de jugadores en todo el mundo. 

\section{Motivación}
%Descripción de un videojuego como software multimedia
Los videojuegos, desde un punto formal, son \textbf{aplicaciones gráficas en tiempo real e interactivas}\cite{libro_esi}. Para que el jugador pueda reaccionar de forma adecuada a los eventos del juego, la aplicación debe de ser capaz de renderizar el entorno del juego con una \textbf{frecuencia} lo suficientemente alta. Si no, el jugador solo percibirá una sucesión de imágenes estáticas. El componente gráfico del videojuego está acompañado de muchos otros sistemas que sirven para recrear el entorno del juego con la máxima precisión posible: simulaciones físicas, sistemas de sonido, inteligencias artificiales, etcétera.

%El desarrollo integra diversas tecnologías
Diseñar un software en el que se integren tantos sistemas de áreas diversas como los que incluye un videojuego \textbf{no es una tarea trivial}. Es necesario diseñar y planificar los distintos sistemas de forma que no haya conflictos o acoplamientos entre ellos. Además, el programa debe estar optimizado de para obtener una frecuencia de refresco de imagen aceptable. La complejidad de esta tarea es tan elevada que muchos estudios utilizan \textbf{motores de juego}, frameworks de desarrollo que suministran gran cantidad de funcionalidad necesaria para el desarrollo de juegos. 

Al desarrollo puramente software del videojuego hay que añadir la elaboración de los \textbf{recursos audiovisuales} que serán utilizados en él. La elaboración de modelos 3D, texturas, músicas, efectos de sonido, animaciones, cinemáticas y demás recursos son procesos que pueden llegar a ser tan complejos como el propio desarrollo software, y requieren de sus propias \textbf{herramientas y tecnologías específicas} que deben ser integradas en el proyecto. 

%Dirección de equipos multidisciplinarios
Ante la enorme dificultad para una sola persona para contar con el conocimiento y la experiencia necesarios en todas las áreas involucradas, en el desarrollo de videojuegos suele realizarse mediante \textbf{equipos de desarrollo}. Dada la naturaleza interdisciplinar de estos equipos, la coordinación es de vital importancia para evitar malentendidos entre profesionales de distintas áreas que puedan desembocar en retrasos en el proyecto.

%Importancia de la experiencia
Los videojuegos cuentan además con un requisito adicional que la diferencia de otros proyectos software: deben resultar \textbf{divertidos para el jugador}. La diversión del jugador es un factor aparte de la usabilidad o la experiencia de usuario que requiere de un buen \textbf{diseño de juego} y de realizar \textbf{pruebas con jugadores} para poder obtener un resultado óptimo.

\section{Descripción del Proyecto}
%Descripción alto nivel del producto y distribución
El objetivo de este proyecto es el diseño y desarrollo de un \textbf{videojuego completo} utilizando el motor de videojuegos Unity3D. El videojuego en cuestión será un juego del género Breakout que se desarrollará en un entorno tridimensional. El producto resultante del proyecto será distribuido a través de Internet mediante una o varias de las plataformas de distribución de juegos. 

%Problemas de diseño de juego que deben abordarse
El juego elegido para ser desarrollado en el proyecto es una \textbf{adaptación a las tres dimensiones del género Breakout}, un género que nació con el juego clásico del mismo nombre \citegame{breakout}. Se trata de un género tradicionalmente 2D, por lo que en esta versión será necesario introducir cambios a las bases de su diseño e implementar una serie de sistemas que mejoren la experiencia del jugador, compensando las dificultades intrínsecas a un entorno 3D (zona de juego más grande, dificultad para apreciar perspectiva...). 

%Implementación software en detalle (Unity, Windows, prestaciones, lenguaje...)
El desarrollo del proyecto se realizará mediante el uso de \textbf{Unity3D}, un \textbf{motor de juegos} multiplataformas enfocado al desarrollo de videojuegos en 2D y 3D. Se trata de un motor de calidad profesional, dotado de renderizado en tiempo real, simulación de físicas (implementado mediante la librería \textbf{PhysX} de Nvidia), tecnología de animación. La funcionalidad central motor está implementada en el lenguaje de programación C++, mientras que la lógica de juego se implementará mediante el lenguaje C\#. El entorno de desarrollo integrado \textbf{Microsoft VisualStudio} se utilizará para la escritura de código, el entorno recomendado para Unity; mientras que la integración de recursos se realizará en el \textbf{editor de Unity}.

%Arte y sonido (blender, photoshop y recursos open source)
Los componentes gráficos del juego se elaborarán mediante el uso de herramientas externas. Los modelos 3D de personajes y otros elementos se modelarán mediante la herramienta de código libre \textbf{Blender}, que permite una exportación de modelos sencilla a los formatos admitidos en Unity. Para la elaboración de las texturas de los modelos, los elementos de la interfaz gráfica de usuario y otras imágenes del proyecto se utilizará el editor gráfico \textbf{Adobe Photoshop}, por su buena relación potencia/simplicidad y por su mayor base de usuarios frente a la competencia, lo que permite obtener soporte con mayor facilidad.

%Control de versiones y distribución.
Durante el desarrollo del proyecto, se llevará a cabo un \textbf{control de versiones} mediante el software \textbf{Git}, utilizando la plataforma \textbf{GitHub} para el almacenamiento de versiones. Para simplificar el uso de Git, se utilizará la herramienta \textbf{SourceTree}, una aplicación que funciona como interfaz gráfica para Git.  Una vez finalizado el proyecto, la distribución de la aplicación resultante se realizará mediante las plataformas en línea de distribución de juegos \textbf{Gamejolt}\footnote{https://gamejolt.com/} y \textbf{Itch.io}\footnote{https://itch.io/
}.
